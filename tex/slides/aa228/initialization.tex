\begin{frame}[fragile]{Belief Initialization}

Before any actions or observations, we start with an initial belief distribution
\begin{itemize}
    \item We can encode prior knowledge in the initial distribution
    \item Generally want to use diffuse (i.e. spread out) initial distributions to avoid over confidence in the absence of information
    \begin{itemize}
        \item In non-parametric representations, a diffuse initial prior may cause difficulties
        \item Thus, we may wait until an informative observation is make to initialize our beliefs
    \end{itemize}
\end{itemize}


\end{frame}




\begin{frame}[fragile]{Example: Landmark Belief Initialization}

\begin{figure}
    \caption{Localization of an autonomous car using a landmark (Example 19.1).}
    \begin{center}
        \begin{jlcode}
            p = let
                Random.seed!(0)

                tikz_string = ""
                tikz_string *= "\\coordinate (ego) at (1,-0.3);\n"
                tikz_string *= "\\draw [rounded corners=1] (\$(ego) - (0.5,0.25)\$) rectangle ++(1,0.5);\n"
                tikz_string *= "\\coordinate (sensor) at (\$(ego) + (0.25,0)\$);\n"
                tikz_string *= "\\def\\ang{12}\n"
                tikz_string *= "\\def\\r{5}\n"
                tikz_string *= "\\coordinate (landmark) at (\$(sensor) + (\\ang:\\r)\$);\n"
                tikz_string *= "\\draw (landmark) circle (0.5);\n"
                tikz_string *= "\\node[anchor=south, font=\\footnotesize] at (\$(landmark) + (0,0.5)\$) {landmark};\n"
                tikz_string *= "\\draw[thick, pastelBlue] (sensor) -- (landmark);\n"
                tikz_string *= "\\fill[pastelBlue] (landmark) circle (0.1);\n\n"

                tikz_string *= "\\node[anchor=south, font=\\tiny] at (\$(sensor)!0.5!(landmark)\$) {\$r\$};"
                tikz_string *= "\\def\\rarc{3}"
                tikz_string *= "\\draw[-{stealth[length=1mm,width=0.5mm]}] (\$(sensor) + (0:\\rarc)\$) arc (0:\\ang:\\rarc);"
                tikz_string *= "\\node[anchor=west, font=\\tiny] at (\$(sensor) + (\\ang/2:\\rarc)\$) {\$\\theta\$};"

                r = 5
                θ = deg2rad(12)
                νr = 0.5
                νθ = 0.1
                Gr = Normal(r, νr)
                Gθ = Normal(θ, νθ)
                Uϕ = Uniform(0.0, 2π)

                landmark_x = 0.5 + r*cos(θ)
                landmark_y = -0.55 + r*sin(θ)

                for i in 1:100
                    r_hat = rand(Gr)
                    θ_hat = rand(Gθ)
                    ϕ_hat = rand(Uϕ)
                    car_x = landmark_x - r_hat * cos(θ_hat)
                    car_y = landmark_y - r_hat * sin(θ_hat)

                    tikz_string *= "\\draw [opacity=0.2, rotate around={$(rad2deg(ϕ_hat)):(landmark)}, rounded corners=1] ($(car_x),$(car_y)) rectangle ++(1,0.5);\n"
                end

                TikzPicture(tikz_string, preamble="", options="x=0.4cm, y=0.4cm")
            end
            plot(p)
        \end{jlcode}
        \plot{fig/sample_from_induced_landmark_obs}
    \end{center}
\end{figure}

{\small Making a range $r$ and bearing $\theta$ observation, we initialize our belief around the landmark.}

\end{frame}