%-------------------------------------------------

\begin{frame}[fragile]{What is a POMDP?}

{\footnotesize
\begin{definitionblock}{POMDP}
    A \textit{Partially observable Markov decision process} (POMDP) is an MDP with \textit{state uncertainty}---meaning we cannot know the \textit{true} state, only a \textit{belief} about the true state using \textit{observations}.
\end{definitionblock}
}

\phantom{}

\begin{itemize}
    \item Formally, a POMDP is defined by the following:
\end{itemize}
\begin{table}[!t]
    {\scriptsize
    \centering
    \caption{\label{tab:solutions} MDP Problem Formulation: $\langle \mathcal{S},\, \mathcal{A},\, {\color{darkblue}\mathcal{O}},\, T,\, R,\, {\color{darkblue}O},\, \gamma \rangle$}
    % \rowcolors{2}{gray!15}{white}
    \begin{threeparttable}
    \begin{tabular}{lll}
        \toprule
        \textbf{Variable} & \textbf{Description} & \textbf{\texttt{POMDPs} Interface} \\
        \midrule
        $\mathcal{S}$ & State space & \texttt{POMDPs.states} \\
        $\mathcal{A}$ & Action space & \texttt{POMDPs.actions} \\
        $\mathcal{O}$ & Observation space & \texttt{POMDPs.observations} \\
        $T(s^\prime \mid s,a)$ & Transition function & \texttt{POMDPs.transition} \\
        $R(s,a)$ & Reward function & \texttt{POMDPs.reward} \\
        $O(o \mid s^\prime)$ & Observation function & \texttt{POMDPs.observation} \\
        $\gamma \in [0,1]$ & Discount factor & \texttt{POMDPs.discount} \\
        \bottomrule
    \end{tabular}
    \end{threeparttable}
    }
\end{table}

\begin{importantblock}
    {\tiny
    \begin{center}
    Remember, a POMDP is a \textit{\textbf{problem formulation}} and \textit{\textbf{not an algorithm}}.
    \end{center}
    }
\end{importantblock}

\end{frame}

%-------------------------------------------------

\begin{frame}[fragile]{How are POMDPs different than MDPs?}

\begin{itemize}
    \item A POMDP\footnote{``Partially observable'' is key in understanding beliefs.} is an MDP with \textit{state uncertainty}
    \begin{align*}
        \text{MDP: }& \langle \mathcal{S},\, \mathcal{A},\, T,\, R,\, \gamma \rangle\\
        \text{POMDP: }& \langle \mathcal{S},\, \mathcal{A},\, {\color{darkblue}\mathcal{O}},\, T,\, R,\, {\color{darkblue}O},\, \gamma \rangle
    \end{align*}
    \item The agent receives an \textit{observation} of the current state rather than the true state (potentially imperfect observations)
    \item Using past observations, the agent builds a \textit{belief} of their underlying state
    \begin{itemize}
        \item Which can be represented by a probability distribution over true states
    \end{itemize}
\end{itemize}

\end{frame}

%-------------------------------------------------

\begin{frame}[fragile]{Example POMDP: Crying Baby Problem}

\begin{columns}[T,onlytextwidth]
    \begin{column}{0.55\columnwidth}
        \begin{figure}
            \centering
            \begin{tikzpicture}
                \node[minimum size=1cm, draw=black, fill=white, circle] (s) {$s_t$};
                \node[minimum size=1cm, draw=black, fill=white, circle, right=0.8cm of s] (s2) {$s_{t+1}$};
                \node[minimum size=1cm, draw=pastelBlue, fill=pastelBlue!40, circle, below=0.5cm of s] (o) {$o_t$};
                \node[minimum size=1cm, draw=black, fill=white, diamond, above=0.5cm of s] (r) {$r_t$};
                \node[minimum size=0.8cm, draw=black, fill=white, rectangle, above=0.5cm of r] (a) {$a_t$};
                \node[left=0.1 of a, anchor=east] {Feed};
                \node[left=0.1 of r, anchor=east] {Reward};
                \node[left=0.1 of s, anchor=east] {Hungry};
                \node[left=0.1 of o, anchor=east] {Crying};
                \draw[->] (s) -- (o);
                \draw[->] (s) -- (s2);
                \draw[->] (s) -- (r);
                \draw[->] (a) -- (r);
                \draw[->] (a) -- (s2);
            \end{tikzpicture}
            \caption{
                \label{fig:crying_baby_pomdp}
                The crying baby POMDP.
            }
        \end{figure}
    \end{column}
    \begin{column}{0.45\columnwidth}
        \begin{itemize}
            \item A simple POMDP with 2 states, 2 actions, and 2 observations:
            \begin{align*}
                \mathcal{S} &= \{\texttt{hungry},\, \texttt{full}\}\\
                \mathcal{A} &= \{\texttt{feed},\, \texttt{ignore}\}\\
                \mathcal{O} &= \{\texttt{crying},\, \texttt{quiet}\}
            \end{align*}
            \item We cannot directly tell if the baby is truly \texttt{hungry}, but we can observe that it's \texttt{crying} and update our \textit{belief} about the true state using this information.
        \end{itemize}
    \end{column}
\end{columns}

\end{frame}

%-------------------------------------------------

\begin{frame}[fragile]{\texttt{QuickPOMDPs}: Crying Baby}

\begin{lrbox}{\cryingbabycode}%
\begin{lstlisting}[language=JuliaLocal, style=julia]
using POMDPs, POMDPModelTools, QuickPOMDPs

@enum State hungry full
@enum Action feed ignore
@enum Observation crying quiet

pomdp = QuickPOMDP(
    states       = [hungry, full],  # 𝒮
    actions      = [feed, ignore],  # 𝒜
    observations = [crying, quiet], # 𝒪
    initialstate = [full], # Deterministic
    discount     = 0.9, # γ

    transition = function T(s, a)
        if a == feed
            return SparseCat([hungry, full], [0, 1])
        elseif s == hungry && a == ignore
            return SparseCat([hungry, full], [1, 0])
        elseif s == full && a == ignore
            return SparseCat([hungry, full], [0.1, 0.9])
        end
    end,

    observation = function O(s, a, s′)
        if s′ == hungry
            return SparseCat([crying, quiet], [0.8, 0.2])
        elseif s′ == full
            return SparseCat([crying, quiet], [0.1, 0.9])
        end
    end,

    reward = (s,a)->(s == hungry ? -10 : 0) + (a == feed ? -5 : 0)
)
\end{lstlisting}
\end{lrbox}%


\begin{columns}[onlytextwidth]
    \begin{column}{0.43\linewidth}
        % print the listing box.
        \scalebox{0.28}{\usebox{\cryingbabycode}}
    \end{column}
    \begin{column}{0.57\linewidth}
        {\footnotesize
        \begin{itemize}
            \item This code$^a$ defines the entire \textit{Crying Baby} POMDP using \texttt{QuickPOMDPs.jl}
            \begin{itemize}
                \item Just a sneak-peek: we'll walk through this in detail in the \texttt{Pluto} notebooks
            \end{itemize}
        \end{itemize}
        }
        \begin{center}
            \scalebox{0.7}{%
            \begin{tikzpicture}
                \node[minimum size=1cm, draw=black, fill=white, circle] (s) {$s_t$};
                \node[minimum size=1cm, draw=black, fill=white, circle, right=0.8cm of s] (s2) {$s_{t+1}$};
                \node[minimum size=1cm, draw=pastelBlue, fill=pastelBlue!40, circle, below=0.5cm of s] (o) {$o_t$};
                \node[minimum size=1cm, draw=black, fill=white, diamond, above=0.5cm of s] (r) {$r_t$};
                \node[minimum size=0.8cm, draw=black, fill=white, rectangle, above=0.5cm of r] (a) {$a_t$};
                \node[left=0.1 of a, anchor=east] {Feed};
                \node[left=0.1 of r, anchor=east] {Reward};
                \node[left=0.1 of s, anchor=east] {Hungry};
                \node[left=0.1 of o, anchor=east] {Crying};
                \draw[->] (s) -- (o);
                \draw[->] (s) -- (s2);
                \draw[->] (s) -- (r);
                \draw[->] (a) -- (r);
                \draw[->] (a) -- (s2);
            \end{tikzpicture}}
        \end{center}
        \footnotetext[1]{Yes, this is self-contained---copy and paste it into a notebook or REPL!}
    \end{column}
\end{columns}

\end{frame}

%-------------------------------------------------

\begin{frame}{POMDP solvers}

A number of ways to solve POMDPs are implemented in the following packages.

\begin{table}[!t]
    {\tiny
    \centering
    \caption{\label{tab:solutions} POMDP Solution Methods}
    \rowcolors{2}{white}{gray!15}
    \begin{threeparttable}
    \begin{tabular}{lcccc}
        \toprule
        \textbf{Package} & \textbf{Online/Offline} & \textbf{State Spaces} & \textbf{Actions Spaces} & \textbf{Observation Spaces}\\
        \midrule
        \href{https://github.com/JuliaPOMDP/QMDP.jl}{\texttt{QMDP.jl}} & Offline & Discrete & Discrete & Discrete \\
        \href{https://github.com/JuliaPOMDP/FIB.jl}{\texttt{FIB.jl}} & Offline & Discrete & Discrete & Discrete \\
        \href{https://github.com/JuliaPOMDP/BeliefGridValueIteration.jl}{\texttt{BeliefGridValueIteration.jl}} & Offline & Discrete & Discrete & Discrete \\
        \href{https://github.com/JuliaPOMDP/SARSOP.jl}{\texttt{SARSOP.jl}} & Offline & Discrete & Discrete & Discrete \\
        \href{https://github.com/JuliaPOMDP/BasicPOMCP.jl}{\texttt{BasicPOMCP.jl}} & Online & Continuous & Discrete & Discrete \\
        \href{https://github.com/JuliaPOMDP/ARDESPOT.jl}{\texttt{ARDESPOT.jl}} & Online & Continuous & Discrete & Discrete \\
        \href{https://github.com/JuliaPOMDP/MCVI.jl}{\texttt{MCVI.jl}} & Offline & Continuous & Discrete & Continuous \\
        \href{https://github.com/JuliaPOMDP/POMDPSolve.jl}{\texttt{POMDPSolve.jl}} & Offline & Discrete & Discrete & Discrete \\
        \href{https://github.com/JuliaPOMDP/IncrementalPruning.jl}{\texttt{IncrementalPruning.jl}} & Offline & Discrete & Discrete & Discrete \\
        \href{https://github.com/JuliaPOMDP/POMCPOW.jl}{\texttt{POMCPOW.jl}} & Online & Continuous & Continuous & Continuous \\
        \href{https://github.com/JuliaPOMDP/AEMS.jl}{\texttt{AEMS.jl}} & Online & Discrete & Discrete & Discrete \\
        \href{https://github.com/JuliaPOMDP/PointBasedValueIteration.jl}{\texttt{PointBasedValueIteration.jl}} & Offline & Discrete & Discrete & Discrete \\
        \bottomrule
    \end{tabular}
    \end{threeparttable}
    }
\end{table}

{\footnotesize
\begin{importantblock}
When defining your problem, the \textbf{\textit{type}} of state, action, and observation space is very important!
\end{importantblock}
}

\end{frame}

%-------------------------------------------------
