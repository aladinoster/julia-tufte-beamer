%-------------------------------------------------

\begin{frame}[fragile]{What is an MDP?}

\begin{definitionblock}{MDP}
    A \textit{Markov decision process} (MDP) is a \textit{problem formulation} that defines how an agent takes sequential \textit{actions} from \textit{states} in its environment, guided by \textit{rewards}---using uncertainty in how it \textit{transitions} from state to state.
\end{definitionblock}

\phantom{}

\begin{itemize}
    \item Formally, an MDP is defined by the following:
\end{itemize}
\begin{table}[!t]
    {\scriptsize
    \centering
    \caption{\label{tab:solutions} MDP Problem Formulation: $\langle \mathcal{S},\, \mathcal{A},\, T,\, R,\, \gamma \rangle$}
    % \rowcolors{2}{gray!15}{white}
    \begin{threeparttable}
    \begin{tabular}{lll}
        \toprule
        \textbf{Variable} & \textbf{Description} & \textbf{\texttt{POMDPs} Interface} \\
        \midrule
        $\mathcal{S}$ & State space & \texttt{POMDPs.states} \\
        $\mathcal{A}$ & Action space & \texttt{POMDPs.actions} \\
        $T(s^\prime \mid s,a)$ & Transition function & \texttt{POMDPs.transition} \\
        $R(s,a)$ & Reward function & \texttt{POMDPs.reward} \\
        $\gamma \in [0,1]$ & Discount factor & \texttt{POMDPs.discount} \\
        \bottomrule
    \end{tabular}
    \end{threeparttable}
    }
\end{table}

\begin{importantblock}
    {\footnotesize
    \begin{center}
    Remember, an MDP is a \textit{\textbf{problem formulation}} and \textit{\textbf{not an algorithm}}.
  
    An MDP formulation enables the use of solution methods, i.e. algorithms.
    \end{center}
    }
\end{importantblock}

\end{frame}

%-------------------------------------------------

\begin{frame}[fragile]{MDP Example: Grid World}

\begin{highlightblock}
    In the \textbf{Grid World} problem, an \textit{agent} moves around a grid attempting to collect as much reward ({\color{julia_green}\textbf{green cells}}) as possible, avoiding negative rewards ({\color{julia_red}\textbf{red cells}}).
\end{highlightblock}

\begin{figure}
    \centering
    \def\svgwidth{\columnwidth}
    \scalebox{0.6}{\input{images/grid-world-states.pdf_tex}}
\end{figure}

\begin{tikzpicture}[remember picture, overlay]
    \node[xshift=-6.18cm, yshift=4.05cm, minimum size=0.2cm, inner sep=0pt, outer sep=0pt, draw=black, fill=yellow, circle] (agent-base) at (current page.south east) {};
    \node[xshift=-2.8cm, yshift=2.3cm, text width=5cm, align=center] (agent-label) at (current page.south east) {Agent's current position\\(i.e., state)};
    \draw[->, dashed] (agent-base) -- (agent-label);
\end{tikzpicture}

\end{frame}

%-------------------------------------------------

\begin{frame}[fragile]{MDP: State space}

\begin{definitionblock}{State space $\mathcal{S}$}

    A set of all possible \textbf{\textit{states}} an agent can be in (discrete or continuous).
\end{definitionblock}

\begin{figure}
    \centering
    \def\svgwidth{\columnwidth}
    \scalebox{0.6}{\input{images/grid-world-states.pdf_tex}}
\end{figure}

\begin{tikzpicture}[remember picture, overlay]
    \node[xshift=3.0cm, yshift=4.0cm, text width=4.2cm, align=center] (state-desc) at (current page.south west) {\textbf{Grid World example}:\\All possible $(x,y)$ cells in a $10\times10$ grid\\(i.e., 100 discrete states)};

    \node[xshift=-6.18cm, yshift=4.05cm, minimum size=0.2cm, inner sep=0pt, outer sep=0pt, draw=black, fill=yellow, circle] (state-base) at (current page.south east) {};
    \node[xshift=-3.2cm, yshift=2.3cm, text width=4cm, align=center] (state-label) at (current page.south east) {\textbf{State}\\$(x,y)$ of $(9,7)$};
    \draw[->, dashed] (state-base) -- (state-label);
\end{tikzpicture}

\end{frame}

%-------------------------------------------------

\begin{frame}[fragile]{MDP: Action space}

\begin{definitionblock}{Action space $\mathcal{A}$}

    A set of all possible \textbf{\textit{actions}} an agent can take (discrete or continuous).
\end{definitionblock}

\begin{figure}
    \hspace{1cm}
    \centering
    \def\svgwidth{\columnwidth}
    \scalebox{0.6}{\input{images/grid-world-actions.pdf_tex}}
\end{figure}

\begin{tikzpicture}[remember picture, overlay]
    \node[xshift=3.0cm, yshift=4.0cm, text width=5cm, align=center] (action-desc) at (current page.south west) {\textbf{Grid World example}:\\The four (discrete) cardinal directions:\\$[\texttt{up}, \texttt{down}, \texttt{left}, \texttt{right}]$};

    \node[xshift=-5.55cm, yshift=2.7cm, minimum size=0.1cm, inner sep=0pt, outer sep=0pt, draw=black, fill=black, circle] (action-base) at (current page.south east) {};
    \node[xshift=-2.5cm, yshift=2.3cm, text width=3cm, align=center] (action-label) at (current page.south east) {Take action \texttt{down} in state $(9,4)$};
    \draw[->, dashed] (action-base) -- (action-label);
\end{tikzpicture}

\end{frame}

%-------------------------------------------------

\begin{frame}[fragile]{MDP: Transition function}

{\scriptsize
\begin{definitionblock}{Transition function\footnotemark[1] $T(s^\prime \mid s, a)$}

    Defines how the agent \textbf{\textit{transitions}} from the current state $s$ to the next state $s^\prime$ when taking action $a$.

    Returns a \textit{\textbf{probability distribution}} over all possible next states $s^\prime$ given $(s,a)$.
\end{definitionblock}
}

\begin{figure}
    \hspace{9.5cm}
    \centering
    \begin{tikzpicture}    
        \draw[step=1cm,gray,very thin] (0,0) grid (3,3);

        \filldraw[fill=julia_blue!70, draw=gray, very thin] (1,2) rectangle (2,3);
        \node[] at (1.5,2.5) {\scriptsize$0.7$};

        \filldraw[fill=julia_blue!10, draw=gray, very thin] (0,1) rectangle (1,2);
        \node[] at (0.5,1.5) {\scriptsize$0.1$};

        \filldraw[fill=julia_blue!10, draw=gray, very thin] (1,0) rectangle (2,1);
        \node[] at (1.5,0.5) {\scriptsize$0.1$};

        \filldraw[fill=julia_blue!10, draw=gray, very thin] (2,1) rectangle (3,2);
        \node[] at (2.5,1.5) {\scriptsize$0.1$};

        \filldraw[fill=white, draw=black] (1,1) rectangle (2,2);
        \node[] at (1.5,1.6) {$\uparrow$};
        \node[] at (1.5,1.3) {$a$};
        \node[] at (1.15,1.85) {\color{gray}$s$};
    \end{tikzpicture}
\end{figure}

\begin{tikzpicture}[remember picture, overlay]
    \node[xshift=6.0cm, yshift=3.5cm, text width=15cm, align=center, font=\footnotesize] (action-desc) at (current page.south west) {\textbf{Grid World example}:\\Stochastic transitions (incorporates randomness/uncertainty).
    \\Action $a{\;=\;}\texttt{up}$ from state $s$.
    \\{\color{julia_blue}70\% chance of transitioning correctly.}
    \\{\color{julia_blue!70}30\% chance (${10\%\times3}$) of transitioning incorrectly.\footnotemark[2]{}}};
\end{tikzpicture}

\footnotetext[1]{Sometimes called the \textit{transition model}.}
\footnotetext[2]{i.e., a different action is taken.}

\end{frame}

%-------------------------------------------------

\begin{frame}[fragile]{MDP: Reward function}

\begin{definitionblock}{Reward function\footnotemark[1]{} $R(s,a)$}

    A defines the \textbf{\textit{reward}} an agent receives when taking action $a$ from state $s$.
\end{definitionblock}

\begin{figure}
    \hspace{5cm}
    \centering
    \def\svgwidth{\columnwidth}
    \scalebox{0.6}{\input{images/grid-world-rewards.pdf_tex}}
    \vspace{-1cm} % NOTE: footnote placement hack
\end{figure}

\begin{tikzpicture}[remember picture, overlay]
    \node[xshift=4.0cm, yshift=3.5cm, text width=8cm, align=center, font=\footnotesize] (action-desc) at (current page.south west) {\textbf{Grid World example}:\\Two cells contain {\color{julia_green}\textbf{positive rewards}}\\and two cells contain {\color{julia_red}\textbf{negative rewards}},\\all others are {\color{black!80}\textbf{zero}}.};
\end{tikzpicture}

\footnotetext[1]{Sometimes called the \textit{reward model}.}

\end{frame}

%-------------------------------------------------

\begin{frame}[fragile]{MDP: Discount factor}

{\scriptsize
\begin{definitionblock}{Discount factor $\gamma \in [0,1]$}

    \item The \textbf{\textit{discount factor}} controls how myopic (short-sighted) the agent is in its decision making (e.g., when $\gamma=0$, the agent only cares about immediate rewards (myopic) and as $\gamma \to 1$, the agent takes in potential future information in its decision making process).
\end{definitionblock}
}

\phantom{}

\captionsetup[sub]{font=tiny, justification=centering}
% NOTE: added [,trim=1cm 0 1cm 0,clip] to the \includegraphics call in the *.pdf_tex files.

\begin{figure}
    \begin{subfigure}[b]{0.24\textwidth}
        \def\svgwidth{\columnwidth}
        \input{images/grid-world-gamma-0.pdf_tex}
        \caption{Short-sighted\\(no reward spread)}
    \end{subfigure}
    \begin{subfigure}[b]{0.24\textwidth}
        \def\svgwidth{\columnwidth}
        \input{images/grid-world-gamma-0.5.pdf_tex}
        \caption{Some future\\reward\footnotemark[1] is spread}
    \end{subfigure}
    \begin{subfigure}[b]{0.24\textwidth}
        \def\svgwidth{\columnwidth}
        \input{images/grid-world-gamma-0.95.pdf_tex}
        \caption{Future reward\\is nicely spread}
    \end{subfigure}
    \begin{subfigure}[b]{0.24\textwidth}
        \def\svgwidth{\columnwidth}
        \input{images/grid-world-gamma-1.0.pdf_tex}
        \caption{Dominated by\\the future reward}
    \end{subfigure}
\end{figure}

\footnotetext[1]{The sum of the \textit{discounted future rewards} is called the \textit{utility} $U(s)$ or the \textit{value} $V(s)$ of a state.}

\end{frame}

%-------------------------------------------------

\begin{frame}[fragile]{\texttt{QuickPOMDPs}: Grid World}

\begin{lrbox}{\gridworldcode}%
\begin{lstlisting}[language=JuliaLocal, style=julia]
using POMDPs, POMDPModelTools, QuickPOMDPs

struct State; x::Int; y::Int end # State definition

@enum Action UP DOWN LEFT RIGHT # Action definition

𝒮 = [[State(x,y) for x=1:10, y=1:10]..., State(-1,-1)] # State-space
𝒜 = [UP, DOWN, LEFT, RIGHT] # Action-space
γ = 0.95 # Discount factor

# Helper for applying actions
const MOVEMENTS = Dict(UP=>State(0,1), DOWN=>State(0,-1), LEFT=>State(-1,0), RIGHT=>State(1,0))
Base.:+(s1::State, s2::State) = State(s1.x + s2.x, s1.y + s2.y)

function T(s, a) # Transition function
    R(s) != 0 && return Deterministic(State(-1,-1))
    Nₐ = length(𝒜)
    next_states = Vector{State}(undef, Nₐ + 1)
    probabilities = zeros(Nₐ + 1)
    for (i, a′) in enumerate(𝒜)
        prob = (a′ == a) ? 0.7 : (1 - 0.7) / (Nₐ - 1)
        destination = s + MOVEMENTS[a′]
        next_states[i+1] = destination
        if 1 ≤ destination.x ≤ 10 && 1 ≤ destination.y ≤ 10
            probabilities[i+1] += prob
        end
    end    
    (next_states[1], probabilities[1]) = (s, 1 - sum(probabilities))
    return SparseCat(next_states, probabilities)
end

function R(s, a=missing) # Reward function
    if s == State(4,3)
        return -10
    elseif s == State(4,6)
        return -5
    elseif s == State(9,3)
        return 10
    elseif s == State(8,8)
        return 3
    end
    return 0
end

abstract type GridWorld <: MDP{State, Action} end

mdp = QuickMDP(GridWorld,
    states     = 𝒮,
    actions    = 𝒜,
    transition = T,
    reward     = R,
    discount   = γ,
    isterminal = s->s==State(-1,-1));
\end{lstlisting}
\end{lrbox}%



\begin{columns}[T,onlytextwidth]
    \begin{column}{0.43\linewidth}
        % print the listing box.
        \scalebox{0.28}{\usebox{\gridworldcode}}
    \end{column}
    \begin{column}{0.57\linewidth}
        {\footnotesize
        \begin{itemize}
            \item This code$^a$ defines the entire \textit{Grid World} problem using \texttt{QuickPOMDPs.jl}
            \begin{itemize}
                \item Just a sneak-peek, as I'll walk through this in more detail in \texttt{Pluto} notebooks
            \end{itemize}
        \end{itemize}
        }
        \begin{figure}
            % \hspace{5cm}
            \centering
            \def\svgwidth{\columnwidth}
            \scalebox{0.7}{\input{images/grid-world-rewards-title.pdf_tex}}
        \end{figure}
        \footnotetext[1]{Yes, this is self-contained---copy and paste it into a notebook or REPL!}
    \end{column}
\end{columns}


\end{frame}

%-------------------------------------------------

\begin{frame}{MDP solvers}

A number of ways to solve MDPs are implemented in the following packages.

\begin{table}[!t]
    {\scriptsize
    \centering
    \caption{\label{tab:solutions} MDP Solution Methods}
    \rowcolors{2}{white}{gray!15}
    \begin{threeparttable}
    \begin{tabular}{lccc}
        \toprule
        % \textbf{Package} & \textbf{Online/Offline} & \textbf{Continuous States} & \textbf{Continuous Actions}\\
        \textbf{Package} & \textbf{Online/Offline} & \textbf{State Spaces} & \textbf{Actions Spaces}\\
        \midrule
        \href{https://github.com/JuliaPOMDP/DiscreteValueIteration.jl}{\texttt{DiscreteValueIteration.jl}} & Offline & Discrete & Discrete \\
        \href{https://github.com/JuliaPOMDP/LocalApproximationValueIteration.jl}{\texttt{LocalApproximationValueIteration.jl}} & Offline & Continuous & Discrete \\
        \href{https://github.com/JuliaPOMDP/GlobalApproximationValueIteration.jl}{\texttt{GlobalApproximationValueIteration.jl}} & Offline & Continuous & Discrete \\
        \href{https://github.com/JuliaPOMDP/MCTS.jl}{\texttt{MCTS.jl}}\tnote{*} & Offline & Continuous & Continuous \\
        \bottomrule
    \end{tabular}
    \begin{tablenotes}
        \scriptsize
        \item[*] {Monte Carlo Tree Search.}
    \end{tablenotes}
    \end{threeparttable}
    }
\end{table}

\begin{importantblock}
When defining your problem, the \textbf{\textit{type}} of state and action space is very important!
\end{importantblock}


\end{frame}

%-------------------------------------------------

\begin{frame}{Reinforcement learning solvers}

{\small Certain problems are better suited in the \textit{reinforcement learning} (RL) domain. Several RL solvers that adhere to the \texttt{POMDPs.jl} interface are implemented in the following packages.}

\begin{table}[!t]
    {\tiny
    \centering
    \caption{\label{tab:solutions} Reinforcement Learning Solution Methods}
    \rowcolors{2}{white}{gray!15}
    \begin{threeparttable}
    \begin{tabular}{lccp{0.5\linewidth}}
        \toprule
        \textbf{Package} & \textbf{State Spaces} & \textbf{Actions Spaces} & \textbf{Algorithms Implemented} \\
        \midrule
        \href{https://github.com/JuliaPOMDP/TabularTDLearning.jl}{\texttt{TabularTDLearning.jl}} & Discrete & Discrete & Q-learning, SARSA, SARSA-$\lambda$ \\
        \href{https://github.com/JuliaPOMDP/DeepQLearning.jl}{\texttt{DeepQLearning.jl}} & Continuous & Discrete & DQN, Double DQN, Dueling DQN, Recurrent Q-learning\\
        \href{https://github.com/ancorso/Crux.jl}{\texttt{Crux.jl}} & Continuous & Continuous & DQN, REINFORCE, PPO, A2C, DDPG, TD3, SAC, Behavior Cloning, GAIL, AdVIL, AdRIL, SQIL, ASAF\\
        \bottomrule
    \end{tabular}
    % \begin{tablenotes}
    %     \tiny
    %     \item[*] {Monte Carlo Tree Search.}
    % \end{tablenotes}
    \end{threeparttable}
    }
\end{table}

{\footnotesize
\begin{importantblock}
When defining your problem, the \textbf{\textit{type}} of state, action, and observation space is very important!
\end{importantblock}
}

\end{frame}

%-------------------------------------------------
